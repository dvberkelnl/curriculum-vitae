\section*{Nevenactiviteiten}

	\begin{subActivityList}
		\item[devnology] Daan is onderdeel van het bestuur van devnology\\
		\url{http://devnology.nl/}\hfill\\

		\item[grunt-peg] Met dit project heeft Daan een
		``itch'' weggenomen. Een missend puzzelstukje tussen
		de JavaScript task runner grunt en het parser
		expression grammar project PEG.js.
		Onlangs is zijn werk gebruikt in de door GitHub
		uitgebrachte Atom editor.\\
		\url{https://www.npmjs.org/package/grunt-peg}\hfill\\

		\item[Topicus University] Om zichzelf en zijn collega's een handvat te
		bieden nieuwe ontwikkelingen te bespreken heeft Daan Topicus University
		opgericht. Eens per maand komen ge\"interesseerden samen om aan de hand
		van een of twee thema's nieuwe inzichten te verwerven. Een overzicht
		van voorgaande avonden is te vinden op\\
		\url{https://sites.google.com/site/topicusuniversity/}.\hfill\\

		\item[tdd-should-be-fun] Op \url{http://sogyotdd.appspot.com/} is een
		Test Driven Development spel te spelen. Daan heeft test scenario's
		geschreven voor deze site. Daarnaast heeft Daan bijgedragen aan de
		uitbreiding van de functionaliteit van de site.\hfill\\

		\item[Weblog] Op \url{http://www.software-innovators.nl/}, de corporate
		weblog van Sogyo, heeft Daan actief geblogged. Op deze blog heeft
		Daan zijn gedachten op het gebied van software ontwikkeling gepubliceerd.
		\hfill\\

		\item[Begeleiding Winnaars] De Radboud Universiteit organiseert
		jaarlijks een wiskunde toernooi voor leerlingen van de middelbare
		school. In 2007 was de hoofdprijs een reis naar New York. De twee
		winnende teams hebben onder begeleiding een trip langs het financi\"ele
		district in New York gemaakt. Daan was \'e\'en van de vier begeleiders.
		\hfill\\

		\item[Essay] ``Trots op mijn studie'' is een bundeling van twaalf essays
		(ISBN: 9789058750884). Daan laat in een van die twaalf essays zien op
		welke wijze en in welk mate hij betrokken is bij de waarden van de
		wetenschap.
		Een licht ingekorte versie van Daan zijn essay is verschenen in het
		``Nieuw Archief voor Wiskunde''.
		\hfill\\
	\end{subActivityList}
